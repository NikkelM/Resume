\addblocktext{Überblick}{
	Ich bin Informatikstudent im Bachelorstudiengang IT-Systems Engineering am Hasso-Plattner-Institut in Potsdam. Fachgebiete die mich besonders begeistern sind unter anderem
	CI/CD, agile Entwicklungsmethoden sowie Machine Learning, wobei ich besonders zu Reinforcement Learning bereits praktische Erfahrungen sammeln konnte. Dies spiegelt sich
	auch im Thema meiner Bachelorarbeit wieder, \emph{Monitoring of Agents for Dynamic Pricing in different Recommerce Markets}.}
%
%Education
\section{Bildung}
\datedexperience{Hasso-Plattner-Institut}{2019-2023}
\explanation{B.Sc., IT-Systems Engineering}
\explanationdetail{Während meines Studiums habe ich mich besonders für Machine Learning und agile Entwicklungsmethoden interessiert. Ich habe jedoch auch in verschiedensten
	anderen Disziplinen Erkenntnisse erlangt, darunter vor allem 3D-Computergrafik, Web-development und Datenbankmanagement.\\
	Bachelorprojektthema: \emph{"Online Marketplace Simulation: A Testbed for Self-Learning Agents"}}
\datedexperience{Europäische Schule Varese}{2014-2019}
\explanation{Europäisches Abitur, Abschluss mit 91.23/100 Punkten}
%
% Experience
\section{Erfahrung}
%
\datedexperience{Norddeutscher Rundfunk}{Sommer 2018 - Kiel}
\explanation{Praktikum}
\explanationdetail{Ich habe unter anderem bei der Planung und Produktion von Beiträgen für die Abendnachrichten assistiert.}
%
\datedexperience{Enterprise Garage Consultancy}{Sommer 2017 - Kalifornien/USA}
\explanation{Praktikum}
\explanationdetail{Ich habe zusammen mit dem Gründer des Unternehmens aus dem Ausland besuchende Unternehmen das "Silicon Valley Mindset" nähergebracht,
	wobei ich in Kontakt mit vielen lokalen Start-Ups gekommen bin. Ich habe außerdem bei der Durchführung von Interviews mit Mitarbeitern aus dfer Automobilbranche unterstützt.}
%
% Skills
\section{Fähigkeiten}
%
\newcommand{\skillprogramming}{\createskill{Programmiersprachen}{
		\textbf{\emph{Vertraut:}} \ \ Python \cpshalf C++ \cpshalf Javascript \ \ }}
%
\newcommand{\skillprogrammingtwo}{\createskill{}{
		\textbf{\emph{Bekannt:}} \ \ Typescript \cpshalf Squeak/Smalltalk \cpshalf Ruby (on Rails)}}
%
\newcommand{\skillsoftdev}{\createskill{Softwareentwicklung}{
		CI/CD \cpshalf Git \cpshalf Agile Methodiken \cpshalf OOP \cpshalf Docker}}
%
\newcommand{\skillframeworks}{\createskill{Frameworks \ \& \ Bibliotheken}{
		Three.js \cpshalf Matplotplib \cpshalf Django \cpshalf Numpy \cpshalf PyTorch}}
%
\newcommand{\skilllanguages}{\createskill{Sprachen}{
		\textbf{\emph{Muttersprache:}} German \ \
		\textbf{\emph{Fließend:}} English \ \
		\textbf{\emph{Fortgeschritten:}} Italian}}
%
\createskills{\skillprogramming, \skillprogrammingtwo, \skillsoftdev, \skillframeworks, \skilllanguages}

\section{Anderes}
\datedexperience{Co-Organisation Eurosport 2019}{2018-2019}
\explanation{Koordinierung von Mitarbeitern, Werbekampagnen und Ergebnissen, Webmaster, Social Media management}
\explanationdetail{Eurosport ist ein in den 1970ern gestarteter zweijährlich stattfindender Wettkampf zwischen den 13 europäischen Schulen.
	2019 hat die Europäische Schule Varese das Event ausgetragen, dem über 800 Schülern aus allen europäischen Mitgliedstaaten beigewohnt haben.}
\datedexperience{Verantwortlichkeit für schulische Videokampagnen}{2015-2019}
\explanation{Image-Film der Schule, Open Day 2019, 2015-2019 End-of-Year Closing Ceremonies}
%
%Footnote
\createfootnote
\end{document}
