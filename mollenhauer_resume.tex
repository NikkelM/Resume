% !TEX root = ./mollenhauer_resume.tex
\documentclass{meta/mollenhauer_cv}

\setname{Nikkel}{Mollenhauer}
\setaddress{Dublin, Ireland}
\setmobile{+49 157 85110381}
\setmail{contact@nikkelm.dev}
\setlinkedinaccount{https://Linkedin.com/in/nikkel-mollenhauer}
\setgithubaccount{http://Github.com/NikkelM}
\setportfolio{https://nikkelm.dev}
\setthemecolor{RedOrange}

\setpicture{0}

\begin{document}

\headerview
\vspace{1ex}

% \addblocktext{Summary}{
% 	I'm a Computer Science student interested in topics such as CI/CD, agile methodology and machine learning, but always open to learning! In autumn of 2022, I joined the Site Reliability team of the Azure cloud platform at Microsoft Dublin, where my responsibilities included extending monitoring functionalities for a number of core Azure services. In late 2022 I launched the 'Random YouTube Video' Chrome extension.}
%
% Experience
\section{Experience}
%
\datedExperienceTitle{Hasso-Plattner-Institute}{Student Assistent}{Potsdam, Germany - 11/2022 - 03/2023}
\explanationdetail{
	In this role I tutored and supported student groups during their software project in relation to the 'Scalable Software Engineering' lecture, with a focus on agile development methodologies. I was also responsible for prototyping an interactive, browser-native tool for interactive exploration of Github statistics using Jupyter notebooks. ({\href{https://github.com/chrisma/RepositoryGuide-Python}{\faGithub}})
}
%
\datedExperienceTitle{Microsoft}{Site Reliability Engineer, Intern}{Dublin, Ireland - 09/2022 - 11/2022}
\explanationdetail{
	I was responsible for creating entirely new monitoring probes covering core Azure services for the synthetic monitoring of the Azure Resource Manager Control Plane, allowing on-call engineers to identify and mitigate incidents that potentially affect millions of customers earlier \& more reliably.
}
%
\datedExperienceTitle{Northern German Broadcasting Company}{Intern}{Kiel, Germany - 07/2018}
% \explanationdetail{
% 	During my internship, I was involved in all steps of the production of the daily evening news, from supporting the filming of reports to live taping.
% }
%
\datedExperienceTitle{Enterprise Garage Consultancy}{Intern}{Palo Alto, California - 06/2017 - 07/2017}
% \explanationdetail{
% 	I conducted and edited interviews with lead engineers working in the automobile sector, aiding research for a book publication. Assisted in introducing the "Silicon Valley Mindset" to company executives from abroad.
% }
%
% Education
\section{Education}
%
\datedexperience{Hasso-Plattner-Institute}{2019 - 2023 - Potsdam, Germany}
\explanation{B.Sc., IT-Systems Engineering, Grade 1.7}
\explanationdetail{
	Bachelor's project: \emph{"Online Marketplace Simulation: A Testbed for Self-Learning Agents"} - Grade 1.0 ({\href{https://github.com/hpi-epic/BP2021}{\faGithub}})\\
	Bachelor's thesis: \emph{"Monitoring of Agents for Dynamic Pricing in different Recommerce Markets"} - Grade 1.3 ({\href{https://github.com/NikkelM/bachelor-thesis}{\faGithub}})}
%
\datedexperience{European School of Varese}{2014 - 2019 - Varese, Italy}
\explanation{European Baccalaureate, Grade 91.23}
%
% Projects
\section{Projects}
\explanation{These are some of my favourite projects. You can find them and all others on my {\href{https://github.com/NikkelM}{\faGithub\ GitHub}} profile.}
%
\datedExperienceTitle{Random YouTube Video}{}{Browser extension}
\explanationdetail{
	A browser extension for Chromium browsers and Firefox that allows users to shuffle truly random videos from any channel. (\href{https://github.com/NikkelM/Random-YouTube-Video}{\faGithub}, \href{https://chrome.google.com/webstore/detail/random-youtube-video/kijgnjhogkjodpakfmhgleobifempckf}{Chrome}, \href{https://addons.mozilla.org/en-US/firefox/addon/random-youtube-video/}{Firefox}, \href{https://microsoftedge.microsoft.com/addons/detail/random-youtube-video/fccfflipicelkilpmgniblpoflkbhdbe}{Edge})\\
	Technologies: Javascript, Webpack, Mocha, Firebase (Realtime Database), Chrome APIs, YouTube Data API.
}
%
\datedExperienceTitle{Outlook Mail Notes}{}{Microsoft Office Add-In}
\explanationdetail{
	The \textbf{Outlook Mail Notes} Add-In allows users to add notes to their e-mails.\\
	Technologies: Javascript, Office API, Quill.js.
}
%
\datedExperienceTitle{JSON-to-Notion, Steam API integration}{}{Notion integrations}
\explanationdetail{
	\textbf{JSON-to-Notion} is a Notion import tool for JSON-formatted data. (\href{https://github.com/NikkelM/JSON-to-Notion}{\faGithub})\\
	The \textbf{Steam API integration} enables importing data from the Steam API to a Notion database. (\href{https://github.com/NikkelM/Notion-Steam-API-Integration}{\faGithub})\\
	Technologies: Javascript, Notion and Steam APIs, JSON schema definition \& validation.
}
% Some extra padding to align the next section
\\
%
% \section{Other}
% %
% \datedexperience{Co-Organization of Eurosport 2019}{2018 - 2019}
% \explanationdetail{Coordination of staff, promotional campaigns \& results, Webmaster, Social Media management}
% %
% \datedexperience{Management of various scholastic Video campaigns}{2015 - 2019}
% \explanationdetail{Image-Movie of the school, Open Day 2019, 2015-2019 End-of-Year Closing Ceremonies}
%
% Skills
\section{Skills}
%
\newcommand{\skillprogramming}{\createskill{Programming Languages}{
		\textbf{\emph{Experienced:}} \ \ Python \cpshalf C\# \cpshalf Javascript \ \ }}
\newcommand{\skillprogrammingtwo}{\createskill{}{
		\textbf{\emph{Familiar:}} \ \ C++ \cpshalf C \cpshalf Ruby (on Rails) \cpshalf Squeak/Smalltalk}}
%
\newcommand{\skillsoftdev}{\createskill{Software Development}{
		CI/CD \cpshalf Azure \& Azure DevOps \cpshalf Git \& GitHub \cpshalf Agile methodology \& Scrum}}
\newcommand{\skillsoftdevtwo}{\createskill{}{
		OOP \cpshalf Docker \cpshalf TTD \cpshalf Automated Pipelines \cpshalf Machine Learning}}
% %
% \newcommand{\skillframeworks}{\createskill{Frameworks \ \& \ Libraries}{
% 		PyTorch \cpshalf Vite \cpshalf OpenTelemetry \cpshalf Django \cpshalf Three.js}}
%
\newcommand{\skilllanguages}{\createskill{Languages}{
		\textbf{\emph{Native}} German \ \
		\textbf{\emph{Fluent:}} English \ \
		\textbf{\emph{Intermediate:}} Italian}}
%
% \createskills{\skillprogramming, \skillprogrammingtwo, \skillsoftdev, \skillsoftdevtwo, \skillframeworks, \skilllanguages}
\createskills{\skillprogramming, \skillprogrammingtwo, \skillsoftdev, \skillsoftdevtwo, \skilllanguages}


%
% Footnote
\createfootnote
\end{document}
